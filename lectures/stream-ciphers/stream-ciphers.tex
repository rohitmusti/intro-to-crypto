

\documentclass{beamer}
 
\usepackage[utf8]{inputenc}
\usepackage{mathtools}

\usetheme{CambridgeUS}
\useoutertheme{split}
\setbeamertemplate{title page}[default][colsep=-4bp,rounded=true]
 
%Information to be included in the title page:
\title{Stream Ciphers}
\author{Rohit Musti}
\institute{CUNY - Hunter College}
\date{\today}
 
\begin{document}
 
\frame{\titlepage}


\begin{frame}
\frametitle{Overview}
\begin{itemize}
    \item We just introduced the concept of semantic security, but referred to it rather obliquely. \pause
    \item While we feel comfortable with its mechanisms, we have not defined an actual cipher that meets its requirements using short keys. 
\end{itemize}
\end{frame}

\begin{frame}
\frametitle{Pseudo-Random Generators (PRG)}
\begin{itemize}
    \item So the primary challenge of using short keys of length \(\ell\) is deriving longer keys that allow us to encrypt messages of length \(L\). \pause 
    \item An PRG can be described as \[ G: \ell \rightarrow L\] \pause
    \item Modified OTP based on a secure PRG: \[ E(s, m) = G(s) \oplus m\] \pause \[ D(s, c) = G(s) \oplus c \]
\end{itemize}
\end{frame}

\begin{frame}
\frametitle{Security of Modified OTP}
\begin{itemize}
    \item The security rests on whether or not \(G(s)\) can be distinguished from \(k\) where \(s \in \{0, 1\}^{\ell}\) and \(k \in \{0, 1\}^{L}\) \pause
    \item What is convenient is that if we prove this to be true, then we can inherit the semantic security from the original OTP! 
\end{itemize}
\end{frame}

\begin{frame}
\frametitle{PRG Security Game}
\begin{itemize}
    \item Let \(S \coloneqq \{0, 1\}^{\ell}\) and \(K \coloneqq \{0, 1\}^{L}\), s.t. \(\ell < L\). Let our adversary \(A\) be computationally bounded. \pause
    \item If experiment 0, challenger samples \(s \xleftarrow[]{R} S\) and generates \(G(s) = k\) and sends to \(\mathcal{A}\). \pause If experiment 1, challenger samples \(k \xleftarrow[]{R} K\) and sends to \(\mathcal{A}\). \pause
    \item \(\mathcal{A}\) returns \(b\), guessing which experiment the challenger conducted. \pause
\end{itemize}

The PRG advantage of \(\mathcal{A}\): \(PRGA[\mathcal{A}, G] = |Pr[W_0] - Pr[W_1]|\) where \(W_b\) is the probability that \(\mathcal{A}\) outputs 1 during experiment \(b\).

\end{frame}

\begin{frame}
\frametitle{Takeaways}
\begin{itemize}
    \item If we have a secure PRG, then we can encrypt messages efficiently given a key shorter than the message. \pause
    \item We don't actually know if PRGs exist. Proving that they exist would demonstrate that \(P = NP\).
    \item This is the bulk of cryptography: we assume that certain problems are hard and use those hardness properties to develop secure protocols.
\end{itemize}
\end{frame}

\begin{frame}
   \frametitle{Content Scrambling System: DVDs}
   \begin{itemize}
       \item Idea: approximate PRG by using Linear Feedback Shift Registers (LFSR) \pause
       \item This a very weak approximation \pause
       \item This disadvantage was compounded by the fact that it was illegal in the United States for manufacturers to export cryptographic systems with keys exceeding 40 bits. \pause
       \item I wonder which US government actor is interested in people not having strong encryption... hmm... beats me ;)
   \end{itemize}
\end{frame}

\begin{frame}
   \frametitle{Breaking CSS}
   \begin{itemize}
   \item Idea: use the stream output bytes and the outputs of one of the smaller LFSRs to obtain the output of another one of the LFSRs. \pause
   \item Mechanism: you brute force guess the smaller LFSR and each guess forces the state of the larger LFSRs, you can check if the output matches to vaildate. \pause
   \item This is better than brute forcing the whole key \(2^{40}\), \pause by 1999 the full key-recovery attack ran in 18 seconds. \pause
   \item Great article in WIRED about this \href{https://www.wired.com/1999/11/why-the-dvd-hack-was-a-cinch/}{click here}
   \end{itemize}
\end{frame}

\begin{frame}
   \frametitle{Other Examples}
   \begin{itemize}
    \item GSM encryption (uses 2 LSFRs), they tried to keep this design private but it was eventually reverse engineered and attacks were found \pause a key lesson here is to never rely on security through obscurity. \pause
    \item The Snowden documents revealed that the NSA can process data encrypted with A5/1 (uses 3 LSFRs) \pause (important to know that there were known attacks on this protocol well before the NSA was known to have cracked it) \pause
    \item Bluetooth E0 (uses 4 LSFRs) has been broken \pause
    \item RC4 cypher, used in SSL/TLS is also broken (though not as badly)
   \end{itemize}
\end{frame}

 
\end{document}
